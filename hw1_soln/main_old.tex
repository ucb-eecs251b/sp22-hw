% Project: hw1
% Author: Erik Anderson
% Date: 09/03/2022

% Document class
\documentclass{article}

% Packages
\usepackage{enumitem}
\usepackage{pdfpages}
\usepackage{hyperref}
\usepackage[margin=1in]{geometry}
\usepackage{placeins}
\usepackage{siunitx}
\usepackage{ltablex}
\usepackage{amsmath}

% Implicit float barriers
\let\Oldsection\section
\renewcommand{\section}{\FloatBarrier\Oldsection}
\let\Oldsubsection\subsection
\renewcommand{\subsection}{\FloatBarrier\Oldsubsection}
\let\Oldsubsubsection\subsubsection
\renewcommand{\subsubsection}{\FloatBarrier\Oldsubsubsection}

% Title Page
\title{Homework 1 Solutions}
\author{EECS 251B}
\date{}

% Document Body
\begin{document}
\maketitle

% Problem 1
\section{System Interconnect}
You have been tasked with building a system-interconnect for a system-on-chip
with 256 processing cores. You are considering several options for the on-chip
system interconnect network: (1) a 16-ary 2-mesh, (2) a 16-ary 2-cube, (3) an
8-ary 2-Cmesh4, and (4) 16-ary 3-fly Clos. The application requires a packet
size of 1024 bits.

\subsection{Calculate the required number of bits per unidirectional}



%\section{First}
%
%\begin{figure}[!hbt]
%    \centering
%    \caption{Example figure}
%    \label{fig:example}
%    \includegraphics[width=0.75\textwidth]{figs/ynwa.png}
%\end{figure}
%
%\section{Second}
%\begin{center}
%    \begin{tabularx}{\textwidth}{|c|c|c|c|}
%        \caption{Example table}\label{table:example}\\
%        \hline
%        Syndrome & Overall Parity & Error Type & Error Code \\ 
%        \hline
%        0 & 0 & No Error & 0b00 \\
%        \hline
%        $\neq0$ & 1 & Single Error & 0b01 \\
%        \hline
%        $\neq0$ & 0 & Double Error & 0b10 \\
%        \hline
%        0 & 1 & Parity Error & 0b11 \\
%        \hline
%    \end{tabularx}
%\end{center}
%
%\section{Third}
%Example equation. Example citation \cite{einstein}. Double citation: \cite{einstein, knuthwebsite}
%\begin{equation}
%    T_{high-low} = \SI{10.119}{\nano\second} - \SI{10.015}{\nano\second} 
%    \approx \SI{104}{\pico\second} 
%\end{equation}
%
%% Start non-bibtex bibliography {9} is for setting hanging indent
%\begin{thebibliography}{9}
%\bibitem{latexcompanion} 
%Michel Goossens, Frank Mittelbach, and Alexander Samarin. 
%\textit{The \LaTeX\ Companion}. 
%Addison-Wesley, Reading, Massachusetts, 1993.
%
%\bibitem{einstein} 
%Albert Einstein. 
%\textit{Zur Elektrodynamik bewegter K{\"o}rper}. (German) 
%[\textit{On the electrodynamics of moving bodies}]. 
%Annalen der Physik, 322(10):891–921, 1905.
%
%\bibitem{knuthwebsite} 
%Knuth: Computers and Typesetting,
%\\\texttt{http://www-cs-faculty.stanford.edu/\~{}uno/abcde.html}
%\end{thebibliography}

\end{document}
