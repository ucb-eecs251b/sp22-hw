% Project: hw1
% Author: Erik Anderson
% Date: 09/03/2022

% Document class
\documentclass{article}

% Packages
\usepackage{enumitem}
\usepackage{pdfpages}
\usepackage{hyperref}
\usepackage[margin=1in]{geometry}
\usepackage{placeins}
\usepackage{siunitx}
\usepackage{ltablex}
\usepackage{amsmath}

% Implicit float barriers
\let\Oldsection\section
\renewcommand{\section}{\FloatBarrier\Oldsection}
\let\Oldsubsection\subsection
\renewcommand{\subsection}{\FloatBarrier\Oldsubsection}
\let\Oldsubsubsection\subsubsection
\renewcommand{\subsubsection}{\FloatBarrier\Oldsubsubsection}

\renewcommand{\thesubsection}{\thesection. \alph{subsection})}
\renewcommand{\thesubsubsection}{\thesection. \alph{subsection}) (\arabic{subsubsection})}

% Title Page
\title{Homework 1 Solutions}
\author{EECS 251B}
\date{}

% Document Body
\begin{document}
\maketitle

\section{System Interconnect}
You have been tasked with building a system-interconnect for a system-on-chip
with 256 processing cores. You are considering several options for the on-chip
system interconnect network: (1) a 16-ary 2-mesh, (2) a 16-ary 2-cube, (3) an
8-ary 2-Cmesh4, and (4) 16-ary 3-fly Clos. The application requires a packet
size of 1024 bits.

\subsection{Calculate the required number of bits per unidirectional channel,
for each network option, such that each of the networks can support the ideal
throughput of 64 bits/cycle/core under uniform random traffic.}

%-------------------------------------------------------------------------------
% Question 1. a). 1).
%-------------------------------------------------------------------------------
\subsubsection{16-ary 2-mesh}
First, let's calculate the total throughput ($\Theta_{total}$) of the network.

\begin{equation}
    \Theta_{total} = N * \Theta_{core} = 256  \text{ cores} * 64  \text{ bits/core/cycle} = 16,384 \text{ bits/cycle}
\end{equation}

Now we need to determine the bisection bandwidth ($B_B$) and the number of
unidirectional bisection channels ($B_C$).  A 4-ary 2-mesh network is shown
below. Let's solve for its $B_B$ and $B_C$ and then generalize the results.

\begin{figure}[!hbt]
    \centering
    \includegraphics[width=0.75\textwidth]{figs/4_ary_2_mesh.png}
\end{figure}

The above figure's bisection cut goes through 4 bidirectional channels. Thus
we get $B_C = 4 * 2 = 8 \text{ unidirectional channels}$. We can also see that under
uniform random traffic this topology should see a bisection bandwidth that
is exactly half of the total throughput. Generalizing our results gives us:

\begin{equation}
    B_C = \sqrt{N} * 2
\end{equation}

\begin{equation}
    B_B = \frac{\Theta_{total}}{2}
\end{equation}

Now we can solve for the unidirectional bandwidth ($b_C$) by dividing the
bisection bandwidth ($B_B$) by the number of bisection channels ($B_C$).
Plugging in the numbers for a 16-ary 2-mesh gives us:

\begin{equation}
    b_C = \frac{B_B}{B_C} = \frac{\frac{16,384}{2}}{16 * 2} = 256 \text{ bits/channel}
\end{equation}
%-------------------------------------------------------------------------------

%-------------------------------------------------------------------------------
% Question 1. a). 2).
%-------------------------------------------------------------------------------
\subsubsection{16-ary 2-cube}
A similar analysis to "Question 1. a) - 16-ary 2-mesh" is performed for this topology.

A smaller network, 4-ary 2-cube, is shown below

\begin{figure}[!hbt]
    \centering
    \includegraphics[width=0.75\textwidth]{figs/4_ary_2_cube.png}
\end{figure}

The above figure's bisection cut goes through 8 bidirectional channels. Thus
we get $B_C = 8 * 2 = 16 \text{ unidirectional channels}$. We can also see that under
uniform random traffic this topology should see a bisection bandwidth that
is exactly half of the total throughput. Generalizing our results gives us:

\begin{equation}
    B_C = \sqrt{N} * 4
\end{equation}

\begin{equation}
    B_B = \frac{\Theta_{total}}{2}
\end{equation}

Now we can solve for the unidirectional bandwidth ($b_C$) by dividing the
bisection bandwidth ($B_B$) by the number of bisection channels ($B_C$).
Plugging in the numbers for a 16-ary 2-mesh gives us:

\begin{equation}
    b_C = \frac{B_B}{B_C} = \frac{\frac{16,384}{2}}{16 * 4} = 128 \text{ bits/channel}
\end{equation}
%-------------------------------------------------------------------------------

%-------------------------------------------------------------------------------
% Question 1. a). 3).
%-------------------------------------------------------------------------------
\subsubsection{8-ary 2-Cmesh4}
A similar analysis to "Question 1. a) - 16-ary 2-mesh" is performed for this topology.

A smaller network, 2-ary 2-Cmesh4, is shown below

\begin{figure}[!hbt]
    \centering
    \includegraphics[width=0.75\textwidth]{figs/2_ary_2_cmesh4.png}
\end{figure}

The above figure's bisection cut goes through 2 bidirectional channels. Thus
we get $B_C = 2 * 2 = 4 \text{ unidirectional channels}$. We can also see that under
uniform random traffic this topology should see a bisection bandwidth that
is exactly half of the total throughput. Generalizing our results gives us:

\begin{equation}
    B_C = \sqrt{\frac{N}{4}}  * 2
\end{equation}

\begin{equation}
    B_B = \frac{\Theta_{total}}{2}
\end{equation}

Now we can solve for the unidirectional bandwidth ($b_C$) by dividing the
bisection bandwidth ($B_B$) by the number of bisection channels ($B_C$).
Plugging in the numbers for a 8-ary 2-Cmesh4 gives us:

\begin{equation}
    b_C = \frac{B_B}{B_C} = \frac{\frac{16,384}{2}}{\sqrt{64} * 2} = 512 \text{ bits/channel}
\end{equation}
%-------------------------------------------------------------------------------

%-------------------------------------------------------------------------------
% Question 1. a). 4).
%-------------------------------------------------------------------------------
\subsubsection{16-ary 3-fly}
A similar analysis to "Question 1. a) - 16-ary 2-mesh" is performed for this topology.

A smaller network, 2-ary 3-fly Clos, is shown below

\begin{figure}[!hbt]
    \centering
    \includegraphics[width=0.75\textwidth]{figs/2_ary_3_fly_clos.png}
\end{figure}

The above figure's bisection cut goes through 4 bidirectional channels. Thus
we get $B_C = 2 * 2 = 8 \text{ unidirectional channels}$. We can also see that under
uniform random traffic this topology should see a bisection bandwidth that
is exactly equal to the total throughput. This is different than the previous topologies.

In the above image, the labeled "1/4" on all  the channels represents the
portion of the total network throughput that any given channel is seeing
under uniform random traffic. Each core inputs 1/4 of the total throughput into
the first stage (because we have 4 cores total in this example). The first
stage switches will split the 1/4 input throughput from one node to two 1/8
throughput streams onto the outgoing channels. This will give us 1/4 of the
total network throughput on both outgoing channels (1/8 from one input and
1/8 from the other). This is repeated until we output 1/4 of the total
network throughput to each of the core's receivers. Now we can draw our
bisection line, shown in red, and calculate the ratio of total throughput
in our bisection cut. It should be very clear that the ratio is 1:1 because
we have 4 channels crossing the bisection each carrying 1/4 of the total
data throughput.

Generalizing our results gives us:

\begin{equation}
    B_C = N
\end{equation}

\begin{equation}
    B_B = \Theta_{total}
\end{equation}

Now we can solve for the unidirectional bandwidth ($b_C$) by dividing the
bisection bandwidth ($B_B$) by the number of bisection channels ($B_C$).
Plugging in the numbers for a 16-ary 3-fly gives us:

\begin{equation}
    b_C = \frac{B_B}{B_C} = \frac{16,384}{256} = 64 \text{ bits/channel}
\end{equation}
%-------------------------------------------------------------------------------

\subsection{Show the worst-case zero-load latency breakdown for each
interconnect network.  Assume the flit size of 64b and shortest (core-to-core)
channel latency of 1 cycle. For Clos, assume that the middle stages are located
at the center of the chip}.

%-------------------------------------------------------------------------------
% Question 1. b). 1).
%-------------------------------------------------------------------------------
\subsubsection{16-ary 2-mesh}
Assuming cut-through flow control, we can calculate the zero-load latency as:

\begin{equation}
    T_{0} = H*t_r + \frac{D}{v} + \frac{L}{b}
\end{equation}

Where $H$ is the hop count, $t_r$ is the router delay, $D$ is the total physical distance
the signal must traverse, $v$ is the speed of the signal, $L$ is the length
of the signal in bits, and $b$ is the channel bandwidth.

We have been given a channel latency of 1 cycle. This latency covers both the
routing delay ($t_r$) and the average time of flight delay for a single hop. We
also know that our serialization delay is the number of cycles required to
serialize the packet across a channel of width $b_C$ (this is the number we
calculated in the previous question). Thus, our latency equation can be
rewritten in terms of "cycles":

\begin{equation}
    T_{0} = H * (\text{1 cycle per hop}) + (\text{flits per packet}) * (\text{cycles per flit})
\end{equation}

The flit size is set to a static 64 bits regardless of the channel width $b_C$.
With a packet size of 1024 bits, this gives us $\frac{1024 \text{ bits per
packet}}{64 \text{ bits per flit}} = 16 \text{ flits per packet}$. "cycles per
flit" is greater than 1 only when the channel width ($b_C$) is less than the
flit size.  The routers will allocate buffer space at the flit level and thus
channels w/ widths greater than the flit size will be underutilized (i.e. flits
are sent sequentially across channels even for channels with widths greater
than the flit size. It is important to note that networks will generally never
be designed such that the flit size is less than the channel width, but this
question is an exception).

We want the worst-case zero-load latency and thus we have to find the
diameter ($H_{max}$) of the network to plug into our latency equation. $H_{max}$ is the
longest minimum-hop path between two nodes. For this topology, $H_{max}$ is found from
one corner node to the opposite corner node. We can now solve for the latency
of the 16-ary 2-mesh:

\begin{equation}
    H_{max} = (\sqrt{N} * 2) - 1 = 31
\end{equation}

\begin{equation}
    T_{0} = H_{max} + (\text{flits per packet}) * (\text{cycles per flit}) = 31 + 16 * 1 = 47 \text{ cycles}
\end{equation}
%-------------------------------------------------------------------------------

%-------------------------------------------------------------------------------
% Question 1. b). 2).
%-------------------------------------------------------------------------------
\subsubsection{16-ary 2-cube}
$H_{max}$ for this topology is found by going from one of the corner middle
nodes to one of the outside corner nodes on the opposite side of the network.
Thus,

\begin{equation}
    H_{max} = \sqrt{N} + 1 = 17 
\end{equation}

\begin{equation}
    T_{0} = H_{max} +  (\text{flits per packet}) * (\text{cycles per flit}) = 17 + 16 * 1 = 33 \text{ cycles}
\end{equation}
%-------------------------------------------------------------------------------

%-------------------------------------------------------------------------------
% Question 1. b). 3).
%-------------------------------------------------------------------------------
\subsubsection{8-ary 2-Cmesh4}
$H_{max}$ for this topology is found by going from one of the corner node
clusters to a node in the opposite corner node cluster.

\begin{equation}
    H_{max} = (\sqrt{N / 4} * 2) - 1 = 15
\end{equation}

\begin{equation}
    T_{0} = H_{max} + (\text{flits per packet}) * (\text{cycles per flit}) = 15 + 16 * 1 = 31 \text{ cycles}
\end{equation}
%-------------------------------------------------------------------------------

%-------------------------------------------------------------------------------
% Question 1. b). 4).
%-------------------------------------------------------------------------------
\subsubsection{16-ary 3-fly}
$H_{max}$ for this topology is found by going from any node to any other node.

\begin{equation}
    H_{max} = 3 
\end{equation}

\begin{equation}
    T_{0} = H_{max} + (\text{flits per packet}) * (\text{cycles per flit}) = 3 + 16 * 1 = 19 \text{ cycles}
\end{equation}
%-------------------------------------------------------------------------------

\includepdf[pages=4-7]{figs/hw1_soln.pdf}

%\section{First}
%
%\begin{figure}[!hbt]
%    \centering
%    \caption{Example figure}
%    \label{fig:example}
%    \includegraphics[width=0.75\textwidth]{figs/ynwa.png}
%\end{figure}
%
%\section{Second}
%\begin{center}
%    \begin{tabularx}{\textwidth}{|c|c|c|c|}
%        \caption{Example table}\label{table:example}\\
%        \hline
%        Syndrome & Overall Parity & Error Type & Error Code \\ 
%        \hline
%        0 & 0 & No Error & 0b00 \\
%        \hline
%        $\neq0$ & 1 & Single Error & 0b01 \\
%        \hline
%        $\neq0$ & 0 & Double Error & 0b10 \\
%        \hline
%        0 & 1 & Parity Error & 0b11 \\
%        \hline
%    \end{tabularx}
%\end{center}
%
%\section{Third}
%Example equation. Example citation \cite{einstein}. Double citation: \cite{einstein, knuthwebsite}
%\begin{equation}
%    T_{high-low} = \SI{10.119}{\nano\second} - \SI{10.015}{\nano\second} 
%    \approx \SI{104}{\pico\second} 
%\end{equation}
%
%% Start non-bibtex bibliography {9} is for setting hanging indent
%\begin{thebibliography}{9}
%\bibitem{latexcompanion} 
%Michel Goossens, Frank Mittelbach, and Alexander Samarin. 
%\textit{The \LaTeX\ Companion}. 
%Addison-Wesley, Reading, Massachusetts, 1993.
%
%\bibitem{einstein} 
%Albert Einstein. 
%\textit{Zur Elektrodynamik bewegter K{\"o}rper}. (German) 
%[\textit{On the electrodynamics of moving bodies}]. 
%Annalen der Physik, 322(10):891–921, 1905.
%
%\bibitem{knuthwebsite} 
%Knuth: Computers and Typesetting,
%\\\texttt{http://www-cs-faculty.stanford.edu/\~{}uno/abcde.html}
%\end{thebibliography}

\end{document}
